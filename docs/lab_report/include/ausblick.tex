% !TEX root = ../report.tex
\chapter{Fazit \& Ausblick}
	\begin{Spacing}{\mylinespace}
	Trotz das auf uns allerlei Probleme zukamen, entstand im Laufe eines Semesters eine Echtzeit Sandkastensimulation, die bereits grundlegende Funktionalität bietet. 
	Das Projekt wurde im  im Laufe des zweiten Semesters grundlegend neu struktuiert und somit anfängliche Performance Probleme erheblich verbessert.
	Nicht nur auf die Erweiterbarkeit (Properties) wurde sehr hohen Wert gelegt sondern es wurden viele neue zusätzliche Parameter unserer Engine hinzugefügt.
	Trotz der Tatsache, das man während der Entwicklung sehr schnell auf viele neue Ideen kommt, die man leider dann doch Aufgrund von Zeitmangel garnicht alle umsetzen kann haben wir uns nicht untekriegen lassen und so viel es ging umgesetzt. Durch diverse DirektX-9 probleme stoßen wir jedoch - wie bereits erwähnt, relativ schnell an die GPU-Grenzen des XNA-Frameworks.
	Eine Neuprogrammierung haben wir jedoch ausgeschlossen - denn es musste eine stabile lauffähige Version bis zum start der Hobit fertig gestellt sein.
	Dieses Projekt beinhaltet sehr viele Möglichkeiten und sollte in der Zukunft fortgeführt werden, jedoch benötigt es einen sehr großen Aufwand an Einarbeitungszeit deswegen wäre es schön wenn zukünftige Fächer der Hochschule bereits Grundlegende Kenntnisse im Bereich der GPU-Programmierung legen könnten.
	
\end{Spacing}
\newpage
\clearpage
%% End Of Doc