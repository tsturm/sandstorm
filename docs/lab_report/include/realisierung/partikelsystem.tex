% !TEX root = ../report.tex
\section{Das Partikelsystem}
\begin{Spacing}{\mylinespace}

Nachdem wir im ersten Projektsemester mit unserem CPU-basierten Partikelsystem sehr schnell an die Grenzen des machbaren gestoßen waren, haben wir uns im zweiten Projektsemester kurzfristig dafür entschieden, das System noch einmal komplett zu überarbeiten und dieses Mal auf eine reine GPU Implementierung zu setzen.

\subsection{Die Anforderungen}

Als Anforderungen haben wir uns gesetzt, ein hoch flexibles und vom restlichen System getrenntes Partikelsystem zu entwickeln, welches die Fähigkeit bietet mehrere hunderttausend oder sogar Millionen von Partikeln in Echtzeit darzustellen.  

\subsection{Die Umsetzung}

\begin{figure}[h!]
	\centering
	\vspace*{30px}
	\includegraphics[width=410px]{graphics/DoubleTargets2.png}
	\caption{Double Rendertargets.}
	\label{fig:RTCahnnels}
\end{figure}

\begin{figure}[h!]
	\centering
	\vspace*{30px}
	\includegraphics[width=410px]{graphics/RendertargetsChannels.png}
	\caption{Kanalbelegung der einzelnen Rendertargets.}
	\label{fig:RTCahnnels}
\end{figure}

\end{Spacing}
\newpage
\clearpage
%% End Of Doc