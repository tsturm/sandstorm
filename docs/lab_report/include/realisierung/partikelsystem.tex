% !TEX root = ../report.tex
\section{Partikelsystem}
\begin{Spacing}{\mylinespace}

Es gibt verschiedene Möglichkeiten ein Partikelsystem zu realisieren.
In unserem Anwendungsfall haben wir uns dafür entschieden Visualisierung und Physik zu trennen.
Dies führt dazu, das es möglich ist, ohne große Abhängigkeiten voneinander parallel zu entwickeln.

\subsection{Die Physik (Physik-Engine)}
Damit die Partikel möglichst realitätsgetreu sich durch die Szene bewegen, benötigt der Computer die Information wie sich die entsprechenden Partikel verhalten sollen.
Dies bedeutet das physikalische Gesetze auf mathematische Funktionen abgebildet werden müssen.
Durch diese Abbildung werden die Bewegungsabläufe der Partikel gesteuert.
Um die Echtzeitfähigkeit des Systems sicher zu stellen müssen meist die Abbildungen (Gesetze) durch ein paar Tricks vereinfacht werden um Rechenkraft zu sparen.

Die Physik-Engine hat genau diese Aufgaben; Sie bewegt die Partikel durch den Raum, erkennt Kollisionen und bildet Gesetzte der \ref{Reflektion} ab.
Durch die hohe Unabhängigkeit von den einzelnen Partikeln eignet sich eine GPU besonders gut zum berechnen entsprechender Gesetze.
Wir haben uns dennoch im aktuellen Projektstatus dazu entschieden die Berechnungen auf der CPU durchzuführen.
Der Grund hierfür liegt in der Problematik das es nicht möglich ist Berechnungen welche auf der GPU stattfinden genauer zu untersuchen bzw. zu Debuggen.

\subsection{Der Renderer (Draw-Engine)}
Der Renderer ist der zweite Teil unseres Partikelsystems. Seine Aufgabe liegt darin, für jedes einzelnes Partikel die Eigenschaften (Farbe,Kraft,Größe,...) für den Benutzer zu visualisieren.
Hierbei wird lediglich lesend auf den vorhandene Datenbestand zugegriffen.
Auch hier ist eine hohe Parallelität möglich, denn jedes Partikel stellt eine unabhängige Einheit dar.
Somit ist es möglich mit Hilfe von DirektX einen Shader für die GPU zu schreiben. 
Dieser erlaubt es, das alle Shader-Units der GPU zusammen an einem Frame arbeiten. 

\end{Spacing}
\newpage
\clearpage
%% End Of Doc