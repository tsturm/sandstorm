% !TEX root = ../report.tex
\section{Die Frameworks}
\begin{Spacing}{\mylinespace}

\begin{figure}[h!]
	\centering
	\includegraphics[width=300px]{graphics/xna.png}
\end{figure}

Für die Interaktion mit der \textit{Kinect}-Kamera und der Darstellung der Landschaft und des Partikelsystems haben wir uns für das \textit{XNA Game Studio 4.0} und das Kinect SDK von \textit{Microsoft} entschieden. Das \textit{XNA Game Studio} ist eine Programmierumgebung die auf \textit{Visual Studio} basiert und zur Entwicklung von Spielen für \textit{Windows-Phone}, \textit{XBox 360} und \textit{Windows}-basierten Computern entworfen wurde. Bestandteil des \textit{XNA Game Studio} ist das \textit{XNA Framework}, welches mehrere auf dem \textit{.Net-Framwork} basierende Bibliotheken vereint und eine sehr einfache und angenehme Schnittstelle zu diesen bereitstellt.
\\\\
Dazu gehören:

\begin{description}
	\item[DirectX] \hfill \\
	DirectX ist eine API für hochperformante Multimedia-Anwendungen und kommt meist bei der Hardware-beschleunigten Darstellung von 2D- und 3D-Grafiken zum Einsatz. 
	\item[XInput] \hfill \\
	XInput ist eine API zur Verarbeitung von Benutzereingaben über Maus, Tastatur und den \textit{XBox 360} Kontroller.
	\item[XACT] \hfill \\
	XACT(Microsoft Cross-Platform Audio Creation Tool) stellt einfache Schnittstellen zur Audiowiedergabe und der Verknüpfung von Sounds an bestimmte Ereignissen bereit.
\end{description}

In unserer Implementierung wird ausschließlich DirectX für die Darstellung und XInput für die Verarbeitung der Benutzereingaben genutzt. Zudem kommt zusätzlich das \textit{Kinect SDK} zur Ansteuerung der \textit{Kinect}-Kamera zum Einsatz, welches auch auf dem \textit{.Net-Framework} basiert und sich dadurch nahtlos und ohne weitere Anpassungen in das System integrieren lässt.

\end{Spacing}
\newpage
\clearpage
%% End Of Doc