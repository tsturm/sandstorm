% !TEX root = ../report.tex

\section{Mathematische Verfahren}

\subsection{Verzerrung von Bildern}

\subsection{Billboarding}

Um unsere Vorgabe der Echtzeitfähigkeit zu erfüllen benötigt es ein paar Tricks, die es erlauben die Komplexität unseres Renderers zu minimieren, gleichzeitig jedoch darf dem Zuschauer diese Manipulation nicht bemerken.
Eine beliebte Technik hierfür ist das Billboarding. Die Idee des Billboardings basiert darauf, komplexe geometrische 3D-Objekte auf ein zweidimensionales Rechteck das sogenannte Billboard runterzubrechen. 
Bei dem Billboard handelt es sich meist um ein vorher berechnetes Bild von dem ursprünglich darzustellenden 3D-Objekts.
Anschließend wird dieses Billboard zur Kamera ausgerichtet, dem Zuschauer fällt es somit sehr schwer zu erkennen, das es sich bei dem gezeigten Objekt um eine zweidimensionale Kopie des 3D-Objektes handelt.
Diese Technik wird hauptsächlich dazu verwendet die benötigten Rechenoperationen für Objekte welche in der Ferne liegen zu minimieren. 
Kommt die Kamera dem tatsächlichen Objekten sehr nahe, wird meist mit einer Interpolation zwischen dem Billboard und dem tatsächlichen 3D-Objekt umgeschaltet.
\end{Spacing}
\newpage
\clearpage
%% End Of Doc