% LaTex Klasse
\documentclass[11pt,a4paper,article,oneside]{memoir}

%% Pakete default
\usepackage[utf8]{inputenc} 
\usepackage[backend=biber,bibencoding=utf8,sorting=nyt,language=ngerman, style=alphabetic]{biblatex}
\usepackage[breaklinks=true]{hyperref}
\usepackage[german]{babel}
\usepackage{graphicx}
\usepackage{subfloat}
\usepackage{eurosym}
\usepackage{amsmath}

%% Pakete usefull
\usepackage{tikz}
%\usepackage{tikzscale}
%\usepackage{bbding}
%\usepackage{listings} 
%\usepackage{epstopdf}
%\usepackage{pgfgantt}
%\usepackage{wrapfig}
%\usepackage{mathtools}
%\usepackage[page]{appendix}

%% R\"omishce zeichen
%\newcommand{\rmnum}[1]{\romannumeral #1}

%%line space
\DisemulatePackage{setspace}
\usepackage{setspace}
%%set global linespace
\newcommand{\mylinespace}{1.1}

% Image floating
%\newcommand{\illustrationname}{Illustration}
%\newfloat[chapter]{illustration}{lol}{\illustrationname}

%% Absatzeinzug!!!
\setlength{\parindent}{0pt}

%% TiKz Stuff %%%
\usetikzlibrary{calc, decorations.pathmorphing,decorations.pathreplacing, fadings, shadings, shapes, arrows,trees, positioning,patterns,automata}
%\pgfdeclarelayer{background}
%\pgfdeclarelayer{foreground}
%\pgfsetlayers{background,main,foreground}
%\usepackage{tikz-uml}
%\usepackage{pgfplots}

%% Bib
%\bibliography{references}
%\defbibheading{bibliography}{\bibsection}


%% PAGE DIMENSIONS A4  % This is from memman.pdf
\settrimmedsize{297mm}{210mm}{*}
\setlength{\trimtop}{0pt}
\setlength{\trimedge}{\stockwidth}
\addtolength{\trimedge}{-\paperwidth}
\settypeblocksize{670pt}{410pt}{*}
\setulmargins{3cm}{*}{*}
\setlrmargins{*}{*}{0.8}
\setmarginnotes{17pt}{51pt}{\onelineskip}
\setheadfoot{\onelineskip}{2\onelineskip}
\setheaderspaces{*}{2\onelineskip}{*}
\checkandfixthelayout

%\maketitle % CUSTOMISATION
 % to be done
 %\pagenumbering{Roman}

% For more than trivial changes, you may as well do it yourself in a titlepage environment 
 %\pretitle{\begin{center}\sffamily\huge\MakeUppercase}
 %\posttitle{\par\end{center}\vskip 0.5em}

%%% ToC (table of contents) APPEARANCE
 %\maxtocdepth{subsection} % include subsections
 %\renewcommand{\cftchapterpagefont}{}
 %\renewcommand{\cftchapterfont}{} % no bold!

%%% HEADERS & FOOTERS
\pagestyle{plain} % try also: empty , plain , headings , ruled , Ruled , companion

%%% CHAPTERS
\chapterstyle{section} %southall} % try also: default , section , hangnum , companion , article, demo

\renewcommand{\chaptitlefont}{\Huge\sffamily\raggedright} % set sans serif chapter title font
\renewcommand{\chapnumfont}{\Huge\sffamily\raggedright} % set sans serif chapter number font

%%% SECTIONS
%\hangsecnum % hang the section numbers into the margin to match \chapterstyle{hangnum}
\maxsecnumdepth{subsection} % number subsections

\setsecheadstyle{\Large\sffamily\raggedright} % set sans serif section font
\setsubsecheadstyle{\large\sffamily\raggedright} % set sans serif subsection font


%% END Memoir customization

%% TikZ input Hack
 \newcommand{\inputTikZ}[1]{\input{#1.tikz}} 
 %\newsubfloat{figure} %% use for \subbottom or \subtop in a figure

%%%##########################################
% Content Composer
\newboolean{de}
\newboolean{fullfrontpage}


\setboolean{de}{true} %DE - true; EN - false
\setboolean{fullfrontpage}{true} % full - true; just a header - false

%set publisher
\newcommand{\docType}{Lab Report}
\newcommand{\docTypeRubric}{Sandkasten Bla Blub Thema}
\newcommand{\docCourse}{Master Projekt System Entwicklung}
\newcommand{\docCourseSemester}{SS 2013}
\newcommand{\docCourseProf}{Prof. Dr. J. Wietzke, Prof. Dr. E. Hergenroether}
\newcommand{\docDate}{01.05.2013}
\newcommand{\docStudentA}{T. Sturm}
\newcommand{\docStudentAMatrikel}{000000}
\newcommand{\docStudentB}{A. Holike}
\newcommand{\docStudentBMatrikel}{724986}
\newcommand{\docStudentC}{S. Arthur}
\newcommand{\docStudentCMatrikel}{000000}
\newcommand{\docStudentD}{M. Djakow}
\newcommand{\docStudentDMatrikel}{000000}

%%% #########################################
%%% BEGIN DOCUMENT

\begin{document}

%Titel
\iffullfrontpage
	% !TEX root = ../report.tex
% begin title
\begin{titlingpage}
	\vspace*{0cm}
	\sffamily 
	\begin{centering}
		\includegraphics[width=0.5\textwidth]{images/fbi_logo.pdf} \\
		\vspace{2.5cm}
		\Huge
			\textbf{\docType} \\
		\vspace{1cm}
		\normalsize
			\docCourse, \docCourseSemester \\
		\small
			(\textit{\docCourseProf}) \\
		\vspace{3cm}
		\LARGE
			\docTypeRubric \\
		\vspace{4cm}
		\normalsize
			\ifde
				vorgelegt von\\
			\else
				submitted by\\
			\fi
		\vspace{1cm}	
		\large
			\docStudentA \hspace{0.1cm} (\docStudentAMatrikel)\\
			\vspace{0.2cm}
			\docStudentB \hspace{0.1cm} (\docStudentBMatrikel)\\
			\vspace{0.2cm}
			\docStudentC \hspace{0.1cm} (\docStudentCMatrikel)\\
			\vspace{0.2cm}
			\docStudentD \hspace{0.1cm} (\docStudentDMatrikel)\\
			\vspace{1cm}
		\normalsize
			\today \\
	\end{centering}
	
\end{titlingpage}

%% End Of Doc
\else
	% !TEX root = ../report.tex
% begin title
\begin{centering}
	\sffamily
	\vspace*{0.5cm}
	\huge
		\docType \\
	\vspace{0.5cm}
	\small
		\docCourse, \docCourseSemester \\
	\footnotesize
		(\textit{\docCourseProf}) \\
	\vspace{0.5cm}
		\large
			\docTypeRubric, \hspace{0.1cm} \docDate \\
		\vspace{0.5cm}
		\small
			\ifde
				vorgelegt von\\
			\else
				submitted by\\
			\fi
		\vspace{0.25cm}	
		\small
			\docStudentA \hspace{0.1cm} (\docStudentAMatrikel)\\
			\docStudentB  \hspace{0.1cm} (\docStudentBMatrikel)\\
		\vspace{0.5cm}
		
		\rule{12cm}{0.025cm}
		
\end{centering}
\vspace{1cm}
%% End Of Doc
\fi

\tableofcontents* % the asterisk means that the contents itself isn't put into the ToC

\clearpage %schreibt letztes Kapitel fertig, noetig fuer toc


%%% BEGIN  CONTENT %%%
%% Styles für TikZ
  %\include{graphics/styles}

% !TEX root = ../report.tex
\chapter{Einleitung}
\begin{Spacing}{\mylinespace}
\begin{figure}[h!]
	\centering
	\includegraphics[width=\textwidth]{graphics/intro.png}
\end{figure}
Die Technik der Strömungssimulation spielt heutzutage eine größere Rolle den je. Lange ist es her das Windkanäle nur zur Erforschung und Verbesserung der Aerodynamik von Flugzeugen genutzt wurde. Heute gibt es kaum noch ein Kraftfahrzeug das nicht im Windkanal optimiert wurde und auch Architekten und Statiker nutzen immer häufiger den Windkanal um ihre Konstruktionen und Berechnungen zu überprüfen. Ein relativ neues Thema in diesem Gebiet ist die Untersuchung kompletter Stadtteilen und Städten im Windkanal. Durch den enormen Anstieg von neuen Wohn-, Gewerbe- und Industriegebieten in den letzten Jahrzehnten, schrumpft der Anteil von freien und natürlichen Flächen immer weiter, wodurch die Frischluftzufuhr negativ beeinflusst wird und sich das Klima in Städten stetig weiter erwärmt und verschlechtert. Um dieser Entwicklung entgegen zu wirken nutzen auch Städteplaner immer häufiger die Vorzüge des Windkanals.
\\\\
So nützlich der Windkanal auch für alle vorgestellten Anwendungen ist, so ist dessen Nutzung auch immer mit sehr hohen Kosten verbunden. Durch die immer weiter steigende Leistung von Computern liegt es deshalb nahe, zu versuchen, erste Tests und Untersuchung von der Realität auf den Computer auszulagern und auf diesem zu simulieren. Genau aus diesem Gebiet bestand der Hauptteil unserer Forschung in diesem Semester. Um dieses doch sehr mathematische und trockene Thema für Außenstehende noch etwas attraktiver und greifbarer zu machen haben wir es jedoch, mit Hilfe der Kinect Kamera von Microsoft, noch um eine Interaktive Komponente erweitert.
\\\\
Auf den Folgenden Seiten werden wir nun unser Vorgehen sowie die Fortschritte, Erfahrungen und Ergebnisse dieses Projekts vorstellen.
\end{Spacing}
\newpage
\clearpage
%% End Of Doc
\clearpage
% !TEX root = ../report.tex
\chapter{Bestehende Arbeiten}
\begin{Spacing}{\mylinespace}

wie wurde das in den usa gemacht\\
was gibt es für ähnliche ansätze\\
beispiele für kinect\\
beispiele für xny\\
beispiele für partikelsystem\\

\end{Spacing}
\newpage
\clearpage
%% End Of Doc
\clearpage
% !TEX root = ../report.tex
\chapter{Konzept}
\begin{Spacing}{\mylinespace}

\section{Modularisierung}
Um eine parallele Entwicklung und spätere Erweiterbarkeit sicherzustellen entschieden wir uns für eine modellbasierte Entwicklung.
Unser Projekt lässt sich in 3 Kategorien gliedern. Das ParticleSystem \textbf{(ParticleSystem)}, die Kinect Ansteuerung \textbf{(SandstormKinect)} und einen Controller \textbf{(Sandstorm)} der Events entgegen nimmt und diese verarbeitet oder ggf. weiterleitet.
Das XNA-Framework unterstützt einen modellbasierten Ansatz indem es jeder DrawableGameCompontent ihren eigenen Kontext zuweist.
Dies bedeutet das jede DrawableGameCompontent ein eigenes kleines Projekt darstellt und unabhängig von den anderen Projekten betrieben werden kann.
\begin{figure}[h!]
	\centering
	\vspace*{20px}
	\includegraphics[width=320px]{graphics/DrawableGame.png}
	\caption{Kompontenten}
	\label{fig:singleColor}
\end{figure}


\end{Spacing}
\newpage
\clearpage
%% End Of Doc
\clearpage
% !TEX root = ../report.tex
\chapter{Grundlagen}
\begin{Spacing}{\mylinespace}

% !TEX root = ../../report.tex
\section{Die Datencontainer}
\begin{Spacing}{\mylinespace}
Um die Echtzeitfähigkeit unseres Systems zu gewährleisten, benötigt es GPU-Unterstützung.
Bevor wir jedoch in technische Details verfallen, werden wir einen kleinen Exkurs machen wie GPUs eigentlich funktionieren.
\subsection{Die GPU}
CPUs und GPUs weisen grundlegend verschiedene Architekturen auf.
\begin{figure}[h!]
	\vspace*{30px}
	\centering
	\includegraphics[width=300px]{graphics/GPUvsCPU.jpg}	
	\caption{GPU-Architektur\protect\footnotemark}
	\label{fig:GPUvsCPU}
\end{figure}
\footnotetext{Quelle: \url{http://www.tomshardware.de/CUDA-Nvidia-CPU-GPU,testberichte-240065-2.html}}
\\
Während eine CPU einen relativ großen Befehlssatz hat um Ganz- oder Fließkommazahlen zu verarbeiten, besitzt hingegen eine GPU einen sehr kleinen Befehlsatz und kann lediglich Fließkommazahlen verarbeiten.
Der große Vorteil einer GPU jedoch ist das sie die Möglichkeit besitzt Berechnungsaufgaben an verschiedene kleinere CO-Prozessoren sogenannte Shader-Units abzugeben.
Durch das zuweisen einer Aufgabe pro Shader-Unit erlaubt eine GPU somit das hoch-parallele abarbeiten von Aufgaben - solange diese unabhängig voneinander sind.
Diese parallele Programmierung hat jedoch auch Nachteile.
Nicht nur das es einer speziellen Programmierung benötigt - sogenannte Shader-Programmierung (Shader-Programme).
Sondern es setzt auch Vorraus, das jede Shader-Unit das gleiche Shader-Programm ausführt.
Besitzen jedoch die zu verarbeitenden Berechnungen genug Unabhänigkeit, so kann eine erhebliche Beschleunigung durch den Einsatz einer GPU welche meist hunderte von Shader-Units besitzt, erzielt werden.


\subsection{GPU-Programmierung}
GPUs und CPUs besitzen unabhängigen Speicher es benötigt somit nicht nur spezieller Programme, sondern auch den schwierigen Teil der GPU Programmierung - den Datentransport zwischen den Speichern.
Zur Vereinfachung nehmen wir in nachfolgenden Kapiteln an, das wir folgendes Viereck (s. Abbildung \ref{fig:Viereck}) zeichnen möchten.

\begin{figure}[h!]
	\vspace*{30px}
	\centering
	\includegraphics[height=130px]{graphics/Quad2.png}	
	\caption{Das Viereck}
	\label{fig:Viereck}
\end{figure}


\subsection{VertexBuffer}
Ein \textit{Vertex Buffer} ist ein Array, welches von dem CPU-Speicher in den GPU-Speicher übetragen wird.
Jeder Array Index bezieht sich dabei auf exakt einen Punkt im Raum.
Bezieht man sich auf unser Viereck aus Abbildung \ref{fig:Viereck} welches wir versuchen zu zeichnen bekommt man also folgendes Array: \\
\begin{figure}[h!]
	\vspace*{30px}
	\centering
	\includegraphics[height=150px]{graphics/vertexbuffer2.png}	
	\caption{Der VertexBuffer}
	\label{fig:VertexBuffer}
\end{figure}
\\
Überträgt man diesen \textit{Vertex Buffer} mittels eines Draw-Calls an die GPU so kann ein minimales Shader-Programm unser übergebenes Viereck zeichnen.

\subsection{IndexBuffer}
Wie man in Abbildung \ref{fig:VertexBuffer} erkennen kann, enthält der \textit{Vertex Buffer} duplizierte Einträge an den Stellen (0-4) und (2-5). Bei dem in unserem Beispiel verwendeten Viereck, fällt dieser unnötige Speicherverbrauch nicht großartig ins Gewicht. Bei weitaus komplexeren Modellen, können diese doppelten Einträge jedoch schnell zu einem Problem werden. Um doppelte Einräge im \textit{Vertex Buffer} zu verhindern, werden sogenannte \textit{Index Buffer} eingesetzt. Diese enthalten Indizes die auf die einzelnen Einträge im \textit{Vertex Buffer} referenzieren. Abbildung \ref{fig:IndexBuffer} zeigt, den für unser Beispiel benötigen, Index- und Vertex Buffer. Wie man sieht sind die doppelten Einträge aus unserem \textit{Vertex Buffer} verschwunden und die Reihenfolge zum Zeichen der beiden Dreiecke, wird nun durch den \textit{Index Buffer} vorgegeben. 
 

\begin{figure}[h!]
	\vspace*{30px}
	\centering
	\includegraphics[height=150px]{graphics/indexbuffer2.png}	
	\caption{Der IndexBuffer}
	\label{fig:IndexBuffer}
\end{figure}

\end{Spacing}
\clearpage
%% End Of Doc
% !TEX root = ../../report.tex

\section{Mathematische Verfahren}

\subsection{Verzerrung von Bildern}

\subsection{Blending}
Blending ist Verfahren welches es ermöglicht überlappende Teilbereiche von zwei Objekten zu definieren.
Im einfachsten Fall - so wie auch in unserem Projekt, kommt das Alpha-Blending zum Einsatz.
Dieses sorgt dafür, das transparente Bereiche zweier Objekte die sich überlappen vermischt werden und somit auch Teile des Hintergrunds sichtbar werden.

\begin{figure}[h!]
	\centering
	\vspace*{30px}
	\includegraphics[width=320px]{graphics/blending.png}	
	\caption{Alpha-Blending\protect\footnotemark}
	\label{fig:AlphaBlending}
\end{figure}
\footnotetext{Quelle: \url{http://common.ziffdavisinternet.com/encyclopedia_images/_ALPHACH.GIF}}



\subsection{Billboarding}

Um unsere Vorgabe der Echtzeitfähigkeit zu erfüllen benötigt es ein paar Tricks, die es erlauben die Komplexität unseres Renderers zu minimieren, gleichzeitig jedoch darf der Zuschauer diese Manipulation nicht bemerken.
Eine beliebte Technik hierfür ist das Billboarding. Die Idee des Billboardings basiert darauf, komplexe geometrische 3D-Objekte auf ein zweidimensionales Rechteck das sogenannte Billboard runterzubrechen. 
Bei dem Billboard handelt es sich meist um ein vorher berechnetes Bild von dem ursprünglich darzustellenden 3D-Objekts.
Anschließend wird dieses Billboard zur Kamera ausgerichtet. Durch den zusätzlichen Einsatz von Blending überlappen diese 2D Objekte Objekte dann und erzeugen somit einen 3D-Effekt. Dem Zuschauer fällt es somit sehr schwer zu erkennen, das es sich bei dem gezeigten Objekt um eine zweidimensionale Kopie des 3D-Objektes handelt.
Diese Technik wird hauptsächlich dazu verwendet die benötigten Rechenoperationen für Objekte welche in der Ferne liegen zu minimieren. 
Kommt die Kamera dem tatsächlichen Objekten sehr nahe, wird meist mittels einer Interpolation zwischen dem Billboard und dem tatsächlichen 3D-Objekt umgeschaltet.
\end{Spacing}
\newpage
\clearpage
%% End Of Doc
\clearpage

Hier kommt immer die Kapitelüberschrift hin, ein kleines Vorgeplänkel was im Kapitel behandelt wird.

%% End Of Doc
\clearpage
% !TEX root = ../../report.tex
\section{Die Kinect}
\begin{Spacing}{\mylinespace}

\subsection{Aufbau und Funktion}

Die Kinect ist als eigenständige Bibliothek konzipiert und kann so in beliebige Projekte integriert werden. Das interne Vorgehen ist in Abbildung \ref{fig:kinectdll} schematisch dargestellt. 

\begin{figure}[hbtp]
	\vspace{15px}
	\centering
	\includegraphics[width=0.4\textwidth]{graphics/block_dll.png}
	\caption{Laboraufbau}
	\label{fig:kinectdll}
\end{figure}

Nach dem Einbinden des Kinect-SDK kann sich ein Objekt der Kinect der Kinect erstellt sowie parametrisiert werden. Diese Objekt bietet nun die Möglichkeit sich auf einen Eventhandler zu registrieren, welcher die Auslieferung der Daten von Seitens der Kinect übernimmt. Dies wird alle in einem separaten Thread ausgeführt um die Leistung des Systems so optimal wie möglich zu halten.

Wird ein Bild empfangen wird, muss es als erstes in einer eigenen Struktur gesichert werden. Anschließend wird es auf die Zielauflösung zugeschnitten.
Dies ist nötig da der Sandkasten eine quadratische Form hat, das Bild jedoch im Format 3:4 (Auflösung von 640x480 Bildpunkte) von der Kinect kommt.
Als nächste Schritt muss der Bildaufbau verändert werden das das Bild um 180 Grad gedreht abgenommen wird. Dies hängt mit dem Hardwareaufbau zusammen, worin die Kinect befestigt ist. Weiterhin müssen die Tiefenwerte angepasst und auf die Höhe des Sandkastens normiert werden. Der Grund hierfür liegt in der späteren Darstellung. Um die Farben der Höhenlinien angenehm verteilt anzeigen zu lassen, ist ein breites Spektrum an Tiefendaten erforderlich. Alle Bilddaten welche nicht im gewünschten Bereich des Sandkastens liegen werden auf unendlich gesetzt.

Abschließend werden die aufbereiteten Daten als Array mit einem Event verschickt und es kann ein weiteres Tiefenbild folgen. Für das Versenden wird deine eigene EventArgument-Klasse verwendet um eine spätere Erweiterung, z.B. hinzufügen von Metadaten, zu erleichtern.

\subsection{Nutzung}

Die Einbindung der Bibliothek ist über einen Verweis zu realisieren. Wenn die erfolgreich geschehen ist kann ein neues Objekt vom Typ "SandstormKinectCore" erstellt werden. Dieses Objekt bietet dann die Registrierung auf einen EventHandler an, welches genutzt werden soll.
Über die öffentlichen Methoden "StartKinect" sowie "StopKinect" kann nun der Betrieb hergestellt werden. Bei jedem neuen Tiefenbild wird über das Event ein Aufbereitetes Tiefenbild versand und kann weiter verwendet werden.

\end{Spacing}
\newpage
\clearpage
%% End Of Doc
\clearpage
% !TEX root = ../report.tex
\section{XNA Renderer}
\begin{Spacing}{\mylinespace}

wie tut der renderer \\
warum haben wir den genommen\\
vorteile\\


\end{Spacing}
\newpage
\clearpage
%% End Of Doc
\clearpage
% !TEX root = ../../report.tex
\section{Das Partikelsystem}
\begin{Spacing}{\mylinespace}

Nachdem wir im ersten Projektsemester mit unserem CPU-basierten Partikelsystem sehr schnell an die Grenzen des machbaren gestoßen waren, haben wir uns im zweiten Projektsemester kurzfristig dafür entschieden, das System noch einmal komplett zu überarbeiten und dieses Mal auf eine reine GPU Implementierung zu setzen.

\subsection{Die Anforderungen}

Als Anforderungen haben wir uns gesetzt, ein hoch flexibles und vom restlichen System getrenntes Partikelsystem zu entwickeln, welches die Fähigkeit bietet mehrere hunderttausend oder sogar Millionen von Partikeln in Echtzeit darzustellen.  

\subsection{Die Umsetzung}

Um unser angestrebtes Ziel zu erreichen, haben wir auf eine Kombination aus verschiedenen Techniken gesetzt, die wir folgend etwas genauer Beschreiben werden.
 
\begin{description}
	\item[Billboarding] \hfill \\
	Für die Darstellung unserer Partikel haben wir zwei unterschiedliche Techniken in Betracht gezogen. Bei der erste und einfacheren Technik wird ein Partikel durch einen einzelnen Vertex repräsentiert und anschließend, als farbiger Punkt, auf den Bildschirm gezeichnet. Da wir aber nicht auf die Möglichkeit verzichten wollten, bei Bedarf unseren Partikeln auch eine Textur zuzuweisen, haben wir uns für die zweite, etwas aufwendigere, Technik das \textit{Billboarding} entschieden.
\\\\
\textit{Billboards} bestehen aus zwei dreieckigen Polygonen die ein Rechteck bilden (s. Abbildung \ref{fig:BBQuad}).	Dieses Rechteck wird anschließend im Vertexshader, mit Hilfe der Viewmatrix der Kamera, so transformiert damit es immer in Richtung des Betrachters ausgerichtet ist. Durch diese Eigenschaft, lassen sich, mit sehr geringem Rechenaufwand, unterschiedlichste Effekte realisieren. In unserem Anwendungsfall, die Darstellung von Rauch beziehungsweise Nebel. 

\begin{figure}[h!]
	\centering
	\vspace*{30px}
	\includegraphics[width=110px]{graphics/billboardQuad.png}
	\caption{Aufbau des Rechtecks für ein Billboard.}
	\label{fig:BBQuad}
\end{figure}
\newpage
	\item[Instancing] \hfill \\
	Bei dieser Technik handelt es sich, um eine von der Grafikhardware bereitgestellten Funktionalität, zur Reduzierung von sogenannten \textit{Drawcalls}. Unter einem \textit{Drawcall} versteht man im Allgemeinen, das Zeichen eines Objekt mit einem bestimmten Material, einer Transformation und gegebenenfalls weiteren Eigenschaften. In unserem Fall wäre also jedes gezeichnete Partikel (Billboard) ein \textit{Drawcall}. Diese sind allerdings sehr teuer und bei der angestrebten Anzahl von über 1.000.000 Partikeln, wäre an eine Echtzeitfähigkeit nicht mehr zu denken gewesen. Somit entschieden wir uns für das \textit{Instancing}. Diese Technik erlaubt es 1.048.576 Instanzen der gleichen Geometrie, in unserem Fall die Billboards, in nur einem einzigen \textit{Drawcall} zu Zeichen. 
	
	\item[Rendertargets] \hfill \\
	In unserer vorherigen CPU-basierten Implementierung des Partikelsystems, war es relativ simpel, die benötigten Eigenschaften (Position, Geschwindigkeit, usw.) unserer Partikel, in einer dazu passenden Datenstruktur im RAM abzulegen und zu manipulieren. Bei der Neuimplementierung musste nun ein Weg gefunden werden, die Speicherung und Manipulation der Partikeleigenschaften, performant auf die GPU zu übertragen.
\\\\
Hierfür entschieden wir uns, für die Nutzung sogenannter \textit{Rendertargets}.   Diese repräsentieren eine spezielle Art von Texturen, für die neben dem standardmäßigen Lesezugriff auch Schreibzugriff zur Verfügung steht. Durch diese Eigenschaft, wird es möglich, \textit{Rendertargets} als Datencontainer zu nutzen und deren Inhalt direkt auf der GPU zu manipulieren.
\\\\
Zusätzlich zu den normalen \textit{Rendertargets}, unterstützt XNA auch \textit{Multiple Rendertargets}. Welche es ermöglichen, während eines Passes, in bis zu vier Rendertargets gleichzeitig zu schreiben. Mit dieser Technik war es uns nun möglich, unsere einzelnen Partikeleigenschaften in den einzelnen Farbkanälen der vier verfügbaren Rendertargets abzulegen (s. Abbildung \ref{fig:RTCahnnels}) und direkt auf der GPU zu manipulieren. 
	
\begin{figure}[h!]
	\centering
	\vspace*{55px}
	\includegraphics[width=410px]{graphics/RendertargetsChannels.png}
	\caption{Kanalbelegung der einzelnen Rendertargets.}
	\label{fig:RTCahnnels}
\end{figure}	

\newpage

	\item[Ping-Pong] \hfill \\
	Im Normalfall können \textit{Rendertargets}, innerhalb eines Passes, entweder nur gelesen oder nur geschrieben werden. Um diese Einschränkung zu umgehen und einen weiteren Pass einzusparen, setzten wir das sogenannte \textit{Ping-Pong} Verfahren ein. Bei diesem Verfahren, existiert von jedem \textit{Rendertarget} ein Duplikat (s. Abbildung \ref{fig:DoubleTarget}). Während eines Passes wird nun eines der Duplikate genutzt um Daten daraus zu lesen und das andere um die manipulierten Daten zurück zuschreiben. Im Anschluss an den Pass, werden die beiden \textit{Rendertargets} dann einfach ausgetauscht. Somit können im nächsten Pass, die zuvor geschrieben Daten gelesen werden und die alten Daten können überschrieben werden.
	
\begin{figure}[h!]
	\centering
	\vspace*{10px}
	\includegraphics[width=410px]{graphics/DoubleTargets2.png}
	\caption{Duplizierte Rendertargets für Ping-Pong Verfahren.}
	\label{fig:DoubleTarget}
\end{figure}

\item[Fullscreen-Pass/Offscreen-Pass] \hfill \\
Das Update unserer \textit{Rendertargets} und somit die Manipulation unserer Partikeleigenschaften, findet in einem sogenannten \textit{Fullscreen-Pass} statt, welcher vor dem eigentlichen Zeichen der \textit{Billboards} durchlaufen wird. Hierbei wird ein im \textit{Screenspace} positioniertes, Bildschirmfüllendes Rechteck genutzt, um ein 1-zu-1 Mapping des zu lesenden und des zu schreibenden \textit{Rendertargets} zu erreichen. Da dieser Pass keine Bildschirmausgabe zur folge hat, kann er auch als \textit{Offscreen-Pass} bezeichnet werden.

	\item[Blending] \hfill \\
Um bestimmte Effekte realistischer Darzustellen, bietet das neue Partikelsystem, optional die Möglichkeit, additives Blending (s. Abbildung \ref{fig:addBlending}) zu aktivieren. Dabei werden die Farben übereinanderliegender Partikel aufsummiert, wodurch Bereiche mit vielen übereinanderliegenden Partikeln heller erscheinen (s. Abbildung \ref{fig:particleResults}).

\begin{figure}[h!]
	\centering
	\vspace*{10px}
	\includegraphics[width=100px]{graphics/additiveBlending.png}
	\caption{Additives Blending.}
	\label{fig:addBlending}
\end{figure}

\end{description}

\newpage
\subsection{Das Ergebnis}

Das Ergebnis unserer Neuimplementierung war mehr als zufriedenstellend. Am Ende hatten wir ein, vom restlichen System getrenntes und sehr flexibles Partikelsystem entwickelt, welches nun komplett auf der GPU lief und uns somit wieder mehr Ressourcen für andere Aufgaben, auf Seiten der CPU zur Verfügung standen. Auch unser, am Anfang noch für utopisch gehaltenes Ziel, eine Anzahl von über 1.000.000 Partikel in Echtzeit darzustellen, konnten wir mit dem neuen System erreichen. Bei über 1.000.000 Partikeln, läuft unsere Anwendung nun immer noch mit über 100 Frames die Sekunde, solche Ergebnisse konnten wir bei der vorherigen CPU-basierten Implementierung, selbst mit minimaler Anzahl an Partikeln nicht erreichen. Abbildung \ref{fig:particleResults} zeigt das Partikelsystem mit einigen unterschiedlichen Konfigurationen.

\begin{figure}[h!]
	\centering
	\vspace*{20px}
	\includegraphics[width=410px]{graphics/ParticleResults.png}
	\caption{Partikelsystem mit unterschiedlichen Konfigurationen.}
	\label{fig:particleResults}
\end{figure}

\end{Spacing}
\newpage
\clearpage
%% End Of Doc
\clearpage
% !TEX root = ../report.tex
\section{Die Benutzeroberfläche}
\begin{Spacing}{\mylinespace}

Nachdem die Grundfunktionalität der Hauptkomponenten unseres Systems standen ging es nun daran eine einfache aber dennoch funktionale Benutzeroberfläche zu entwerfen. Da das \textit{XNA-Framework} von Haus aus auf \textit{Windows-Forms} zur Darstellung von Benutzeroberflächen setzt, beschlossen auch wir vorerst diese Variante zu nutzen. Hielten uns aber die Möglichkeit offen eventuell später auf das etwas modernere \textit{WPF}-System zu wechseln.
\\\\
Hauptanforderungen waren ein übersichtliches Design und ein einfaches Hinzufügen von neuen Funktionalitäten. Um diese Anforderungen zu erfüllen entschieden wir uns für eine schlichte Statusleiste am unteren Rand des Editor-Fensters für einfache Anzeigen wie zum Beispiel die Frames Pro Sekunde(FPS) oder die Anzahl der Partikel und ein Tab-Panel an der rechten Seite des Editor-Fensters zur Konfiguration der einzelnen Komponenten. Durch die Nutzung des Tab-Panel lässt sich eine gute Separierung der einzelnen Komponenten in der Benutzeroberfläche realisieren. 
\\\\
Die Kommunikation zwischen der Benutzeroberfläche und den einzelnen Komponenten ist über das im \textit{.Net-Framework} integrierte Event-System realisiert. Bei einer Interaktion mit der Benutzeroberfläche wird ein entsprechendes Event gefeuert, welches anschließend die benötigten Daten an alle Komponenten liefert, die sich zuvor für dieses Event registriert haben.  

\begin{figure}[h!]
	\vspace*{30px}
	\includegraphics[width=\columnwidth]{graphics/gui.png}	
	\caption{Die GUI}
	\label{fig:GUI}
\end{figure}

\end{Spacing}
\newpage


\section{Reflection}
Im Laufe der Zeit wurde unser Projekt immer größer, dies brachte auch viele neue Funktionalitäten mit sich.
All diese neuen Features mussten wir stetig unserer GUI-Oberfläche hinzufügen. Dieser sehr statische Ansatz wurde deshalb durch Reflektion in einen dynamischen überführt.
Diverse moderne Programmiersprachen so auch unser verwendetes C-Sharp besitzen die Möglichkeit während des Programmablaufs Informationen über die Struktur eines gegebenen Objekts abzurufen.

\begin{figure}[h!]
	\vspace*{30px}
	\includegraphics[width=\columnwidth]{graphics/reflection.png}	
	\caption{Reflection}
	\label{fig:Reflection}
\end{figure}

Dieser Ansatz und die Tatsache das wir diverse Probleme mit unserer statischen Multi-Window GUI hatten, haben uns dazu bewegt unser GUI-System auf ein Reflection basiertes dynamisch erzeugtes Single-Window GUI-System umzusteigen.
\clearpage
%% End Of Doc
\clearpage
% !TEX root = ../report.tex
\chapter{Zusammenfassung}
\begin{Spacing}{\mylinespace}

hier schreiben wir unsere erfahrungen rein undwas wir genau hinbekommen haben. zudem sollen probleme die währed der arbeit aufgetreten sind erwähnt / erläutert werden. \\

\end{Spacing}
\newpage
\clearpage
%% End Of Doc
\clearpage
% !TEX root = ../report.tex
\chapter{Probleme}
	\section{Echtzeitfähigkeit}
		Leider besitzt die derzeitige Ausarbeitung diverse kleinere Probleme, welche die Echtzeitfähigkeit des Systems gefährden.
		Diverse teile von Berechnungen werden noch wie in \ref{physik} beschrieben auf der CPU ausgeführt, während der Teil der Visualisierung bereits auf die GPU portiert wurde.
		Dies führt zu erheblichen Performanceproblemen, denn es muss bei jeder Physikberechnung (jeden Frame), die Partikeldaten zwischen GPU und CPU kopiert und synchronisiert werden.
	\section{Darstellung}
		Die Darstellung stellte sich um Laufe des Projektes als schwieriger heraus als vorher angedacht.
		Hierbei kann man die Probleme auf welche wir gestoßen sind grob in Hard- und Softwareprobleme unterscheiden.
		\subsection{Hardware}
			Trotz das wir einen Beamer von einem Grafiklabor der Hochschule zur Verfügung gestellt bekommen haben, bemerkten wir bereits bei ersten Tests, das ein großer Farbunterschied zwischen Beamer und
			Monitor vorhanden ist. Leider scheint das Spektrum unseres Beamers sehr begrenzt zu sein, so das wir einen Farbunterschied zwischen weiß und gelb kaum wahrnehmen können.		
		\subsection{Software}
			Durch die physikalische Gegebenheit das Kinekt und Beamer sich an unterschiedlichen Orten befinden, entsteht bei der Projektion zusätzlich zur Verzerrung auch noch das Problem der Verschiebung.
			Die Kalibrierung stellte sich somit schwieriger heraus als bisher gedacht, deshalb wurden aus zeitlichen Gründen der Fokus auf Aufgaben gesetzt um schnellstmöglich eine lauffähige Version zu erstellen.

\chapter{Ausblick}
	\begin{Spacing}{\mylinespace}
	Trotz das auf uns allerlei Probleme zukamen, entstand im Laufe eines Semesters eine Echtzeit Sandkastensimulation, die bereits grundlegende Funktionalität bietet. 
	Im Laufe des nächsten Semesters werden wir dann Aufgaben, welche in diesem Semester ein wenig vernachlässigt wurden wie z. B. die Kalibrierung nachbessern.
	Des Weiteren werden wir die bisherigen Physikberechnungen auf die GPU portieren um so hoffentlich wieder die Echtzeitfähigkeit des Systems zu erlangen.
	Auch neue Funktionalitäten sind geplant, welche notwendig sind um unser eigentliches \ref{Ziel} zu erreichen.
	
\end{Spacing}
\newpage
\clearpage
%% End Of Doc
\clearpage
 %and so on
\inputTikZ{graphics/hardware}
%% end of doc
\clearpage

%\pagenumbering{Alph}

%\newpage
%\printbibliography

%\newpage
%\listoffigures
%\newpage

%%%Anhang
%\appendix
%\renewcommand{\appendixtocname}{Anhang}
%\renewcommand{\appendixpagename}{\textsf{Anhang}}

%\appendixpage

\end{document}
